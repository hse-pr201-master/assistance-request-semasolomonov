\documentclass[a4paper,11pt]{article} 

\usepackage{geometry}
\geometry{top=20mm}
\geometry{bottom=15mm}
\geometry{left=20mm}
\geometry{right=20mm}
\renewcommand{\baselinestretch}{1.5}
\usepackage{indentfirst}

\usepackage[flushleft]{threeparttable}
\usepackage{mathtext} 	
\usepackage{graphicx}
\usepackage{float}
\usepackage{wrapfig}
\usepackage{amsmath}
\usepackage{indentfirst}
\usepackage{lastpage}
\usepackage{tikz}
\usepackage{pgfplots}
\usepackage{braket}
\usepackage{verbatim}
\usetikzlibrary{arrows}
\usetikzlibrary{calc,positioning,fit,backgrounds}
\usepackage{amsmath} 
\usepackage[T2A]{fontenc}
\usepackage[utf8]{inputenc}	
\usepackage[english,russian]{babel}
\usepackage{amssymb}
\usepackage{icomma} 
\usepackage{mathtext} 
\usepackage{mathrsfs}
\usepackage{mathtools}
\usepackage{hyperref}
\usepackage{mathtext} 
\usepackage{multicol}
\usepackage{tikz,lipsum,lmodern}
\usepackage[most]{tcolorbox}
\usepackage{lipsum}
\usetikzlibrary{arrows}
\tcbuselibrary{breakable}
\usepackage{hyperref}
\hypersetup{%
	colorlinks = true,
	linkcolor  = black
}
\makeatletter
\renewcommand{\maketitle}{\begin{center}
		\noindent{\LARGE\@title}\par
		\noindent {\large\itshape\@author}
		\noindent {\large\@date}
		\vskip 2ex\end{center}}
\makeatother
\title{
	Мотивационное письмо \\ 
	\begin{large} 
		\textcolor{red}{НИУ ВШЭ. Теория вероятностей и статистика.}
\end{large} }
\date{Сезон 2019-2020}
\begin{document}
	\maketitle
	
	\begin{enumerate}
		\item \textcolor{blue}{ФИО:} Соломонов Семен Феликсович
		\item \textcolor{blue}{Номер группы:} БЭК183
		\item \textcolor{blue}{Оценка за курс:} На момент подачи заявления (без учета экзамена по мат.статистике) -- 8 за первый семестр и 7 за второй
		\item \textcolor{blue}{Уровень владения красотами программирования:}
			\begin{itemize}
			 	\item \textit{Phyton:} Постарался взять максимум из курса <<Наука о данных>> во втором семестре
			 	\item \textit{R:} Бувально недавно начал осваивать, но пока в рамках курса, например,  написал (хоть и не очень красивые, на мой взгляд) скрипты для доверительных интервалов и гипотез для какой-никакой тренировки
			 	\item \textit{Stata:} Можно еще отметить это: начинающий, работал со Статой при обработке данных в курсовой работе
			 \end{itemize}
		 \item \textcolor{blue}{Уровень владения \LaTeX:} Хороший, начал им пользоваться на регулярной основе в конце 1-го курса
		 \item \textcolor{blue}{Мои контакты:}
		 	\begin{itemize}
		 		\item \textit{Телефон:} +7 (905) 443-90-29
		 		\item \textit{Почта:} solomonovsemen@gmail.com
		 		\item \textit{Телеграмм:} @semasolomonov
		 	\end{itemize}
	\end{enumerate}

	\begin{tcolorbox}[enhanced jigsaw,breakable,pad at break*=1mm, attach boxed title to top center={yshift=-3mm,yshifttext=-1mm},
	colback=blue!5!white,colframe=blue!100!black,colbacktitle=blue!100!black,
	title=Почему именно меня должны взять в ассисенты?,fonttitle=\bfseries,
	boxed title style={size=small,colframe=black!100!black}]
	С самого начала 2-го года обучения ТВиС стал для меня самым интересным курсом, ему я уделял большее количество своего времени. Не могу не сказать, что огромное влияние оказали наш любимый лектор и семинарист Елена Владимировна, а также Штаб (Илья) за что выражаю им благодарность за крутое изложение материала и атмосферу, эти люди дали мне понять, что ТВиС -- это не просто дисциплина с кучей формальностей, а что-то большее. Именно поэтому я отношусь к этому курсу особенно. Самое главное для меня удовольствие -- это не только получить знания, но и объяснить их максимально подробно (как это было с моими одногруппниками). Мне очень нравится изучать что-то новое за рамками программы, конечно же, в ключе теории вероятностей, поэтому студенты не останутся без интересных вещей и интуитивного понимания предмета.
	
	Да, я не студент ИП, но я настроен бороться и смогу показать, что я тоже что-то умею \textbf{:)}
	\end{tcolorbox}

	\newpage
	\begin{center}
		{\large \textcolor{red}{Задача:} <<Что лучше лифт или лестница?>>}
	\end{center}

	Предыстория: У нового здания ВШЭ на Покровке есть один минус -- лифты. Некоторые студенты их ждут относительно долго и поэтому могут спуститься по лестнице, но не делают этого, потому что мы в Вышке ленивые. Давайте упрощенно узнаем, а оправданно ли студенты ждут лифт при определенных условиях? Будем рассматривать случай, когда существует две альтернативы: спуститься до нужного этажа по лестнице или ожидать лифт (числа взяты из примерных соображений, никаких эмпирических подсчетов не было). Кейс таков: до пары остается ровно 3 минуты, студент находится на 6-ом этаже корпуса R, лекция по теории вероятностей в R201, 2 минуты на то, чтобы добраться до двери и 1 для того, чтобы найти место и сесть в аудитории.
	
	Пусть случайная величина $ X $ -- время ожидания лифта (в минутах) имеет равномерное распределение на отрезке от 0 до 5, то есть $ X \sim U[0;5] $. Пусть лифт двигается быстро и время на спуск мы не учитываем. Количество времени, за которое можно спуститься по лестнице (в минутах) есть случайная величина $ Y $, обладающее произвольным распределением со следующей функцией плотности: $ f(y) = \begin{cases}
	\frac{2}{9}y, \text{ } \text{при} \text{ } y\in[0;3] \\
	0, \text{ } \text{при} \text{ } x\notin[0;3]
	\end{cases} $ (принцип построения: чем больше времени имеется на спуск по лестнице, тем с большей вероятностью можно добраться до нужного этажа).
	
	\begin{enumerate}
		\item [(a)] Найти вероятность ожидания лифта более чем 2 минуты
		\item [(b)] Найти вероятность спуститься до нужного этажа за 2 минуты
		\item [(c)] На основе подсчетов в (а) и (b) сделать вывод
	\end{enumerate}

	\begin{center}
		{\large Решение:}
	\end{center}

	\begin{enumerate}
		\item [(a)] 
		Из общего вида функции плотности равномерного распределения получаем, что
		\[
		f(x) = 	\begin{cases}
		\frac{1}{5}, \text{ } \text{при} \text{ } x\in[0;5] \\ 
		0, \text{ } \text{при} \text{ } x\notin[0;5]
		\end{cases} 
		\]
		Тогда, $ \mathbb{P}(X\in[2;5]) =  \Large \displaystyle \int\limits_{2}^{5}\dfrac{1}{5}dx = \dfrac{1}{5}x\bigg|^{5}_{2}=\dfrac{3}{5}=0.6$
		
		\item [(b)] Покажу, что функция плотности построена хорошо и условие нормировки выполнено: 
		
		\[ 
		\Large \displaystyle \int\limits_{0}^{3}\dfrac{2}{9}ydy+\Large \displaystyle \int\limits_{0}^{3}0dy = \dfrac{2y^2}{18}+0=\dfrac{y^2}{9}\bigg|_{0}^{3}=\dfrac{9}{9}=1
		\]
		
		Аналогично п.(а), при заданной плотности: $ \mathbb{P}(Y\in[0;2]) = \Large \displaystyle \int\limits_{0}^{2}\dfrac{2y}{9}dy = \dfrac{y^2}{9}\bigg|_{0}^{2}=\dfrac{4}{9}=0.(4)$
		
		\item [(c)] Из п.(а) видим, что вероятность добраться до аудитории (с учетом времени спуска на лифте, которая очень маленькая) за 2 минуты равна $ 1 - 0.6 = 0.4 $ (40\%), а вероятность дойти до двери пешочком примерно 44\%. Так как с большей вероятностью студент успеет добежать до аудитории, спускаясь по лестнице, лучше лифт не ждать. 
	\end{enumerate}
\end{document}